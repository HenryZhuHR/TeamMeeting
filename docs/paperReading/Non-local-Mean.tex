\section{图像去噪算法: Non-Local Mean 非局部均值滤波}



\begin{frame}{Non-local Means 非局部均值滤波}
    原始图像$v$的每个像素点$v(x)$经过滤波之后可以得到对应图像上的点$\tilde{u}(x)$
    \begin{equation}
        \tilde{u}(x)=\sum_{y\in \Omega_x} w(x,y)v(y)
        \label{nl-mean}
    \end{equation}    
    公式\ref{nl-mean}中,$w(x,y)$是一个权重,表示在原始图像$v$中,像素$x$、$y$之间的相似性。$\Omega_x$是像素$x$的邻域。同时权重需要满足公式\ref{nl-mean-weight_condition}的条件
    \begin{equation}
        w(x,y)>0\quad and\quad \sum_{y\in \Omega_x} w(x,y)=1,\forall x\in \Omega,y\in \Omega_x
        \label{nl-mean-weight_condition}
    \end{equation}

    对于每一个像素,其去噪后的结果等于其邻域中像素$y$的加权和,权重则是像素$x$、$y$之间的相似性。该邻域称为“搜索邻域”,其范围越大,可以找到相似的概率也就越大大,但是计算量也会增大。
\end{frame}

\begin{frame}{Non-local Means 非局部均值滤波}
    用欧氏距离来衡量两个像素之间的相似性
    \begin{equation}
        w(x,y)=\frac{1}{n(x)} \exp(\frac{||V(x)-V(y)||_{2,a}^2}{h^2})
    \end{equation}
    $n(x)$是归一化因子,是所有权重和,使得满足公式\ref{nl-mean-weight_condition}中权重和为1的条件。

    $V(x)$和$V(y)$分别是像素$x$和$y$的块(Patch)邻域,小于搜索邻域。
    $||V(x)-V(y)||_{2,a}^2$是两个邻域的的\textbf{高斯加权欧氏距离}\footnote{求欧式距离的时候,距离块邻域的中心越近的权重越大,越远则越小,权重服从高斯分布,因此需要在距离上乘以一个高斯权重。},高斯权重为:    
    \begin{equation}
        G(x,y)=\frac{1}{2\pi a^2}\exp(-\frac{(x^2+y^2)}{2a^2})
    \end{equation}
\end{frame}


\begin{frame}{Non-local filter}
    \begin{figure}
        \centering
        \includegraphics[width=0.65\textwidth]{docs/paperReading/Non-local/non-local-means.png}
    \end{figure}
\end{frame}

