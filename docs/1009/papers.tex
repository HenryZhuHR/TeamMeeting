\section{Paper Reading: HVS in Adversarial Example}

\begin{frame}{HVS(Human Visual System)产生对抗样本的相关论文}
    \begin{itemize}
        \item 1. \emph{The Human Visual System and Adversarial AI(2020)}: HVS对低频信息更敏感,对亮度的变化比色度的变化更敏感
        \item 2. \emph{SSIMLayer: Towards Robust Deep Representation Learning via Nonlinear Structural Similarity(2018)}: 结构相似性度量 structural similarity metric, 一个提取结构信息的神经网络模块(模仿HVS)
        \item 3. \emph{Demiguise Attack: Crafting Invisible Semantic Adversarial Perturbations with Perceptual Similarity}: 利用感知相似度(一种新的图像质量度量指标,可以模拟真实世界中光照和对比度变化)来产生扰动的黑盒攻击,使用面向hvs的图像度量来处理语义信息,以生成不可见的语义对抗扰动。可以作为一个部分融合到传统攻击方法中
    \end{itemize}
\end{frame}

\begin{frame}{HVS(Human Visual System)产生对抗样本的相关论文}
    \begin{itemize}
        \item 4. \emph{GreedyFool: Multi-Factor Imperceptibility and Its Application to Designing Black-box Adversarial Example Attack}: 根据影响人眼可感知性的因素(显著畸变(JND)、韦伯-费希纳定律、纹理掩蔽和信道调制)设计多因素度量损失产生对抗样本
        \item 5. \emph{CDAE: Color decomposition-based adversarial examples for screen devices}: 为屏幕设备设计的基于颜色分解的对抗性示例方法DAE
        \item 6. \emph{Semantic Adversarial Examples}: 语义对抗样本,约束优化问题,在HSV色彩空间上添加扰动(应该基于HVS对色度变化不敏感的特点)
        \item 7. \emph{Feature Distillation: DNN-Oriented JPEG Compression Against Adversarial Examples}: 基于图像压缩技术的抗对抗实例攻击方法
    \end{itemize}
\end{frame}

\begin{frame}{影响HVS的因素}
    \emph{The Human Visual System and Adversarial AI}
    \begin{itemize}
        \item \textbf{HVS对低频信息更敏感}
        \item \textbf{HVS对亮度的变化比色度的变化更敏感。}
    \end{itemize}

    \emph{GreedyFool: Multi-Factor Imperceptibility and Its Application to Designing Black-box Adversarial Example Attack}
    \begin{itemize}
        \item \textbf{Just Noticeable Distortion}: JND。人眼无法感受像素周围的明显低于失真阈值以下的刺激 
        \item \textbf{Weber-Fechner Law}: 一个心理物理学的观点,明显的刺激差异保持一个恒定的比率
        \item \textbf{Texture Masking}: 纹理掩膜。人眼对平滑区域像素的干扰比纹理区域的干扰更敏感(也就是对低频变化比高频变化更加敏感)
        \item \textbf{Channel Modulation}: 通道调制,人眼对颜色通道的敏感度是有差异的。对绿色最敏感,对蓝色最不敏感。
    \end{itemize}
\end{frame}











\section{Paper Reading: IQA in Adversarial Example}

\begin{frame}{IQA( Image Quality Assessment)产生对抗样本的相关论文}
    \begin{itemize}
        \item 1.\emph{Feature Distillation: DNN-Oriented JPEG Compression Against Adversarial Examples}: 基于图像压缩技术的抗对抗实例攻击方法
        \item 2.\emph{RAN4IQA: Restorative Adversarial Nets for No-Reference Image Quality Assessment}: 基于GAN的无参考IQA
        \item 3.\emph{VR IQA NET: Deep Virtual Reality Image Quality Assessment using Adversarial Learning}: 将对抗学习应用到VR IQA中
        \item 4.\emph{Generating Adversarial Examples with an Optimized Quality}: 直接利用IQA的指标来产生对抗样本
        \item 5.\emph{A Novel Rank Learning Based No-Reference Image Quality Assessment Method}
    \end{itemize}
\end{frame}